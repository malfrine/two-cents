%! Author = mdas
%! Date = 12/7/2020

% Preamble
\documentclass[11pt]{article}

% Packages
\usepackage{amsmath,amsfonts}
\usepackage{bbm}

\title{Two Cents - Pennies Formulation}
\begin{document}
\maketitle
\section{Variables and Parameters}
\subsection{Sets}
\begin{itemize}
    \item $m \in m$: set of months in the planning horizon = \{0, 1, 2, ...\}
    \item $t \in T_w$: set of decision periods when user is working = \{0, 1, 2, ... N \}
    \item $t \in T_r$: set of spending periods when user is retired = \{N + 1, N + 2, ... N + M \}
    \item $i \in I$: set of instruments = \{student loan, mutual fund rrsp, line of credit, ...\}
    \item $l \in L$: set of loans = \{student loan, line of credit, ...\}
    \item $l \in RL$: set of revolving loans = \{line of credit, ...\}
    \item $j \in INV$: set of investments = \{mutual fund rrsp, gic, ...\}
    \item $e \in E$: set of entities that tax people on income = \{federal, provincial \}
    \item $b \in B_{e}$: set of tax brackets associated with a given taxing entity = \{federal: \{bracket 1, bracket 2, ...\}, provincial: \{bracket 1, bracket 2, ... \}\}
\end{itemize}


\subsection{Parameters}
\begin{itemize}
    \item $r_{i, t}$: average interest/growth rate of instrument $i$ in decision period $t$, $i \in I, t \in T_w$
    \item $v^{0}_{i}$: starting balance of instrument $i$, $i \in I$
    \item $\alpha$: user risk tolerance fraction $\in$ [0, 1]
    \item $A$: monthly allowance
    \item $a^{min}_{i, t}$: min allocation of instrument $i$ in decision period $t$, $i \in I, t \in T_w$
    \item $t^{f}_{i}$: final decision period of instrument $i$, $i \in I$
    \item ${U}_{l}$: upper bound value of loan $l$, $l \in L$
    \item $c^{v}$: constraint violation penalty cost
    \item $t^{f}$: final pament horizon
    \item $n_{t}$: number of months in payment horizon $t$, $t \in T_w$
    \item $\gamma^{min}$: minimum investment volatility of portfolio
    \item $\gamma^{max}$: maximum investment volatility of portfolio
    \item $s_{t}$: monthly salary for user during decision period $t$, $t \in T_w \cup T_r$
    \item $m^{tax}_{e, b}$: marginal tax rate for entity $e$ in tax bracket $b$, $b \in B_e, e \in E$
    \item $s^{cut}_{e, b}$: cutoff salary corresponding to tax for entity $e$ in tax bracket $b$, $b \in B_e, e \in E$
\end{itemize}

\subsection{Decision Variables}
\begin{itemize}
    \item $x_{i, t}$: allocation towards instrument $i$ in decision period $t$, $i \in I, t \in T_w$
    \item $w_{i, t}$: withdrawal from investment $i$ in spending period $t$, $i \in INV, t \in T_r$
    \item $s_{t}$ allocation slack during decision period $t$, $t \in T_w$
\end{itemize}

\subsection{Auxilliary Variables}
\begin{itemize}
    \item $v_{i, t}$: balance of instrument $i$ at the start of decision period $t$, $i \in I, t \in T_w$
    \item $b^{unpaid}_{l, t}$: 1 if loan $l$ is not paid off in pament period $t$, $l \in L, t \in T_w$
    \item $b^{debt}_{t}$: 1 if user is in debt during pament period $t$, $t \in T_w$
    \item $y^{r_i}_{t}$ investment risk violation during decision period $t$, $t \in T_w$
    \item $y^{r}_{t}$ total risk violation during decision period $t$, $t \in T_w$
    \item $y^{l}_{l, t}$ loan due date violation during decision period $t$, $l \in L, t \in T_w$
    \item $\gamma^{lim}_t$ volatility limit during decision period $t$, $t \in T_w$
    \item $s^{tax}_{t}$ taxable salary during decision period $t$, $t \in T_w \cup T_r$
    \item $p^{tax}_{t, e, b}$ amount of payable taxes accrued during decision period $t$ to entity $e$ under income bracket $b$. $t \in T_w \cup T_r, e \in E, b \in B_e$
\end{itemize}


\section{Workload Variance Minimization Model}
\subsection{Objective Function}
\begin{equation}
    \max \sum_{i \in I} v_{i, t^f}
          + \sum_{i \in I, t \in T} v_{i, t} r_{i, t}
          - 0.01 c^v \sum_{t \in T} y^{r_i}_{t} + y^{r}_{t}
          - 0.1 c^v \sum_{l \in L, t \in T} b^{l}_{l, t}
          - \sum_{t \in T_w \cup T_r, e \in E, b \in B_e} p^{tax}_{t, e, b}
\end{equation}

\subsection{Constraints}
\begin{align}
    v_{l, t} \geq U_{l} b^{unpaid}_{l, t} & \qquad \forall l \in L , t < t^f \in T_w \\
    -v^{0}_{l} r_{l, t} \leq x_{l, t} + A (1 - b^{unpaid}_{l, t+1}) & \qquad \forall l \in RL , t < t^f \in T_w \\
    a^{min}_{l, t} \leq x_{l, t} + A (1 - b^{unpaid}_{l, t+1}) & \qquad \forall l \in L | l \notin RL  , t < t^f \in T_w \\
    a^{min}_{i, t} \leq x_{i, t} & \qquad \forall i \in INV, t \in T_w \\
    v^{0}{i} = v_{i, 0} & \qquad \forall i \in I \\
    v_{i, t} = v_{i, t-1} (1 + r_{i, t-1})^{n_t} + x_{i, t} \frac{(1 + r_{i, t-1})^{n_t-1}} {r_{i, t-1}} & \qquad \forall i \in I, t \in T_w| t > 0  \\
    \sum_{i \in I} x_{i, t} + s_t = A & \qquad \forall t \in T_w \\
    v_{l, t} \leq 0 & \qquad \forall l \in L, t \in T_w \\
    y^{l}_{l, t} \geq -v_{l, t} & \qquad \forall t \in T_w| t > t^f_l - 1, l \in L  \\
    b^{debt}_{t} \geq \frac{ \sum_{l \in L} b^{unpaid}_{l, t} } { |L| } & \qquad \forall t \in T_w \\
    \gamma^{lim}_t = \gamma^{min} (1 - b^{debt}_t) + \alpha (\gamma^{max} - \gamma^{min} (1 - b^{debt}_t)) & \qquad \forall t \in T_w \\
    y^{r}_{t} \geq \sum_{i \in I} x_{i, t} v_{i} - A \gamma^{lim}_t & \qquad \forall t \in T_w \\
    y^{r_i}_{t} \geq \sum_{i \in INV} x_{i, t} v_{i} - \gamma^{lim}_t \sum_{i \in INV} x_{i, t} & \qquad \forall t \in T
\end{align}

\begin{enumerate}
\setcounter{enumi}{0}
    \item The objective is to maximize final net worth and interest earned while minimizing risk and load due date violation costs
    \item The definition for the loan unpaid indicator
    \item Pay the minimum monthly payment for revolving loans if the loan is not paid off
    \item Pay the minimum monthly payment for instalment loans if the loan is not paid off
    \item Pay the pre-authorized monthly contribution for investments always
    \item Define the starting balance of all instruments
    \item Define the interest gained for an instrument over the decision period. This constraint is actually different for instruments (e.g. cash) where the interest rate is 0
    \item Sum of all allocations must equal the allocation limit
    \item Loans cannot have a positive balance
    \item The loan due date violation must be greater than absolute value of the loan balance
    \item Define an indicator to determine if the user is in debt for a given decision period
    \item Define the volatility limit as a linear interpolation between the max and min volatilities. If the user is in debt, the minimum volatility is 0.
    \item Define the total risk violation
    \item Define the investment risk violation
\end{enumerate}


\end{document}
